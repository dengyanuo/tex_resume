\input ../header/example_tex01.tex

% \FFrf \baselineskip = 14pt
  \FFrg \baselineskip = 15pt

{
James.Y.Deng
}

{ 
% \FFrb
2090 Pauline Blvd \#2B, Ann Arbor, MI 48103
*
(408)444-5520
*
deng.ya.nuo@gmail.com
}

{ \smallbreak } 

{\par\noindent\hrule} 

{ \bigbreak } 



% https://careers.hpe.com/job/Hewlett-Packard-Enterprise-Roseville-California/142793795
% Embedded Software Engineer
% Roseville, California
AMD , 
Boxborough Massachusetts 01719(Remote)


{ \bigbreak } 
{ 
%Dear Shawn Taylor,
Dear Human Resource Manager, 
}

{ \bigbreak } 
I am applying for the position of `` 
Verification Engineer - Senior
'' offered by AMD .

{ \bigbreak } 
I have two first-pass tapeout SOC experience.
One is in 2006, for a dual-MIPS cores SOC. 
Another is in 2019, for a small PIC mcu intergrated with analog/power control circuits.
I use NCsim + Verdi, which include DPI/PLI.
I have written DPI/PLI module(.so in Linux) by myself.
I lived in Ann Arbor now. I have a long term plan to move to Kentucky for a peace life.
So I want a find a remote work.

{ \bigbreak } 
I got my Master degree in Semiconductor ( exact : Micro Electronics And Solid State Electronics) in 2005.
Then I worked as a digital IC designer for Vbridge in San Jose, California for 3 years.
It's a design house, who first tapeout a dual-MIPS SOC chip in TSMC with 130nm successfully.
I wrote all the IP cores for the security engine of the SOC chip and were verificated succefully.
As you know, the FPGA is the base tool to verify ASIC before tapeout.
I used Xilinx FPGA and Lattice FPGA at that time.
After the chip was tapped out, I worked in the software group to boot it up from zero.
It's a dual-MIPS chip, one runs Linux and another runs uCos.
I debuged the boot code by UART to boot up Linux \& uCos from single flash chip,
and built the Board Support Package(BSP).
Later, the team was dismissed and I started to work as an Embedded System Engineer.
In 2016 I worked in Blueway and contributed to the team in control logic using Altera FPGA.
In 2021, I was hired as a Senior Electric Engineer(software) by H3D, 
who manufactures the first class nuclear radiation detectors.
However, when I start working there, I was requested to work as
a hardware engineer and was involved in schematics , layout PCB, and debug electric boards.
Besides, I have a list of low priority R\&D tasks , 
including optimizing the gcc for ARM and FPGA in sample ADC channel,
and writing Linux driver for usb-C.

{ \bigbreak } 
I am NOT only an ASIC IC design engineer, but an embedded / FPGA software engineer. 
I can co-operate to work with the linux/RTOS software team and the PCB team.
Moreover, I can write IP cores for the tapeout of an ASIC SOC chip if necessary.

{ \bigbreak } 
I sincerely belive that my background will really meet your expectation.
I am looking forward to elaborating on how my specific skills and abilities will benefit AMD .

{ \bigbreak } 
I can be contacted at deng.ya.nuo@gmail.com
or (408)444-5520 to arrange for a convenient time to meet.

{ \bigbreak } 




\vskip 60pt

{ \bigbreak } 
Sincerely,

James.Y.Deng

\bye
