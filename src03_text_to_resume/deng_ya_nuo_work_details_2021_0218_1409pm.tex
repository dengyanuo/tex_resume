\special{papersize=8.5in,11.5in}
% 0.4 padding 
\hsize 7.7 in     % 8.5  - 0.4 * 2
\vsize 10.7 in    % 11.5 - 0.4 * 2
\voffset -0.6 in  % 1 - 0.4
\hoffset -0.6 in  % 1 - 0.4

% \hsize 6.9 in     % 8.5  - 0.8 * 2
% \vsize 9.9 in     % 11.5 - 0.8 * 2
% \special{papersize=8.5in,11.5in}
% % 0.4 padding 
% \voffset -0.2 in  % 1 - 0.8
% \hoffset -0.2 in  % 1 - 0.8

\nopagenumbers

\parindent = 0 pt
\emergencystretch = 800 em
%\tolerance=10000
%\baselineskip = 16pt
\baselineskip = 12pt


\font\FFaa=      cmb10            at 24pt
\font\FFbb=      cmr10            at 8.5pt
\font\FFcc=      cmb10            at 13pt
\font\FFdd=      cmb10            at 13pt
\font\FFee=      cmti10           at 13pt


\bigbreak

\centerline{ \FFaa
How to answer the questions in the interview.
}

{ \medbreak } { \FFcc
What's IP cores?
} { \smallbreak } {\par\noindent\hrule} { \smallbreak }

    The first company I worked for was vbridge Microsyste.
It's a design house. Its proudce were verified IP cores.
7 new IP cores were designed for the security engine : 
AES, DES, 3DES, MD5, SHA1, CRC, and a 5 bits MCU.
I finished all IP cores at a rate of one month one IP core.

The design were include specification, user manual, testbench, 
RTL codes, timeing check script. 
Each IP core have about 10 modules and 3000 to 5000 lines.
3 to 4 of them were testbechs, and the others are RTLs.


{ \medbreak } { \FFcc
How to finish a IP core?
} { \smallbreak } {\par\noindent\hrule} { \smallbreak }

The proceduce are : 

{ \parindent = 1 em \item{1} 
using VIM to create or modify the codes;
}
{ \parindent = 1 em \item{2} 
using NC to simulate the RTL code;
}
{ \parindent = 1 em \item{3} 
using perl to analyze the simulation result;
}
{ \parindent = 1 em \item{4} 
using verdi to debug the code if necessary.
}
{ \parindent = 1 em \item{5} 
using VIM to create or modify the timing rules;
}
{ \parindent = 1 em \item{6} 
using synopsys DC to generate the netlist.
}
{ \parindent = 1 em \item{7} 
using NC to simulate the netlist;
}
{ \parindent = 1 em \item{8} 
using perl to analyze the simulation result again.
}

If ok, the testbench, the netlist, and the timing rules can be sent to the backend.



{ \medbreak } { \FFcc
How to teach the new person?
} { \smallbreak } {\par\noindent\hrule} { \smallbreak }

{ \parindent = 1 em \item{1} 
Using git / svn / cvs to control your code ; write enough meaningful comment when save each version.
}
{ \parindent = 1 em \item{2} 
Read the specification carefully.
}
{ \parindent = 1 em \item{3} 
Write the testbench and generated the objective data; double check the specification before start to write RTL codes.
}
{ \parindent = 1 em \item{4} 
The manpower is expensive. Using more script to compare the data automaticlly. 
}
{ \parindent = 1 em \item{5} 
Prefer using more scripts than using wave debuger to find the bug 
unless in the critical module(s).
}

\bye
