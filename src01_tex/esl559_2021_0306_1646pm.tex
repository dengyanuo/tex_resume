\input ../header/example_tex01.tex

\parskip 1.5 em
\FFrk \baselineskip = 22pt



% \vfil
% \break %%%%%%%%%%%%%%%%%%% page break
% 
\parskip 0.5 em
% %\FFrh \baselineskip = 16pt
% \FFri \baselineskip = 18pt
% \FFrj \baselineskip = 18pt
\FFri \baselineskip = 17pt


\par 1. Why is LinkedIn a great place to start your job search?
\par 2. What is the best way to find a job on LinkedIn? Why?
\par 3. What is the second best way to find a job on LinkedIn? Why?
\par 4. How many people are actively using LinkedIn? How many active jobs are there on LinkedIn?
\par 5. Why is it important to complete the "Career Interests" section of your profile on LinkedIn?
\par 6. Why should you write a personalized message to potential employers and not just copy your cover letter on LinkedIn?
\par 7. How can LinkedIn Groups help you connect with potential employers?
\par 8. What are the LinkedIn Job Alerts? Why should you use them?
\par 9. How can you use LinkedIn if you are currently unemployed? How will you show up in the LinkedIn search results? What should you do?
\par 10. Why should you focus more on your skills and not so much on your job titles when using LinkedIn?

\medbreak

\par LinkedIn Profile Tips - Questions
\par 1. What kind of photo should you use for your LinkedIn Profile? Why should you not use an everyday photo of you and your friends?
\par 2. What should you include in your profile title? The video mentions adding keywords to your profile title. Why should we include keywords? If you are using LinkedIn to find an accounting job, what kind of keywords can we use?
\par 3. What two things are going to entice someone to visit your LinkedIn profile?
\par 4. What is a cover image on your profile? What are two strategies video mentions to make your cover image stand out?
\par 5. What is your profile summary? Why it is important? What are the three things that your LinkedIn profile should do?
\par 6. How can you find the actual keywords that potential employers may use to find you on LinkedIn? Give an example of how to do this.
\par 7. Why is it important to not include too many skills on your LinkedIn profile? How can you find the best skills to list?

1. You should use a professional photo or a working photo.
Because you are hunting a job, you should be more professional.

2. The current or last work's title.
The employers will search, and the Linkedin will matched by the keywords.
All the keywords should be related to the target position "accounting".

3.
How much are your skills and your experences matched the target position.

4.
The cover image on the Linkedin profile should be a 1584 (w) x 396 (h) pixels png/jpg file ,
which related to your professional work.
First, related to your professional work ; 
Second, unique and instering.

5.
The profile summary is about how about your past experences and what's your new target position.
That help the employer(s) to match and decide whether you are the qualified candidate.
Your skills, your experences, your target positon.

6.
Your skills being used in your professional work would be the best keywords in your profile to attract the employers.
For example : if you want a firmware engineer position, you can add a keywork "linux driver developer" into your profile.
The reason is : the employer knows the software engineer with "linux driver developer" is very experenced and can do most works in linux world.

7.
The Linkedin allows max 32 keywords.
In most cases 16 keywords is good.
All the important skills that will be used in your professional work should be shown.


\bye
