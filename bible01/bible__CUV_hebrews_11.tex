\input ../header/example_tex01.tex
\FFrh \baselineskip = 13pt
\parskip 0.3 em


\centerline{  \FFbg
希 伯 來 書 11 - Chinese Union Version (Traditional) }

\par
11 
信就是所望之事的實底,是未見之事的確據。
\par
2 
古人在這信上得了美好的證據。
\par
3 
我們因著信,就知道諸世界是藉神話造成的;這樣,所看見的,並不是從顯然之物造出來的。
\par
4 
亞伯因著信,獻祭與神,比該隱所獻的更美,因此便得了稱義的見證,就是神指他禮物作的見證。他雖然死了,卻因這信,仍舊說話。
\par
5 
以諾因著信,被接去,不至於見死,人也找不著他,因為神已經把他接去了;只是他被接去以先,已經得了神喜悅他的明證。
\par
6 
人非有信,就不能得神的喜悅;因為到神面前來的人必須信有神,且信他賞賜那尋求他的人。
\par
7 
挪亞因著信,既蒙神指示他未見的事,動了敬畏的心,預備了一隻方舟,使他全家得救。因此就定了那世代的罪,自己也承受了那從信而來的義。
\par
8 
亞伯拉罕因著信,蒙召的時候就遵命出去,往將來要得為業的地方去;出去的時候,還不知往那裡去。
\par
9 
他因著信,就在所應許之地作客,好像在異地居住帳棚,與那同蒙一個應許的以撒、雅各一樣。
\par
10 
因為他等候那座有根基的城,就是神所經營所建造的。
\par
11 
因著信,連撒拉自己,雖然過了生育的歲數,還能懷孕,因他以為那應許他的是可信的。
\par
12 
所以從一個彷彿已死的人就生出子孫,如同天上的星那樣眾多,海邊的沙那樣無數。
\par
13 
這些人都是存著信心死的,並沒有得著所應許的;卻從遠處望見,且歡喜迎接,又承認自己在世上是客旅,是寄居的。
\par
14 
說這樣話的人是表明自己要找一個家鄉。
\par
15 
他們若想念所離開的家鄉,還有可以回去的機會。
\par
16 
他們卻羨慕一個更美的家鄉,就是在天上的。所以神被稱為他們的神,並不以為恥,因為他已經給他們預備了一座城。
\par
17 
亞伯拉罕因著信,被試驗的時候,就把以撒獻上;這便是那歡喜領受應許的,將自己獨生的兒子獻上。
\par
18 
論到這兒子,曾有話說:從以撒生的才要稱為你的後裔。
\par
19 
他以為神還能叫人從死裡復活;他也彷彿從死中得回他的兒子來。
\par
20 
以撒因著信,就指著將來的事給雅各、以掃祝福。
\par
21 
雅各因著信,臨死的時候,給約瑟的兩個兒子各自祝福,扶著杖頭敬拜神。
\par
22 
約瑟因著信,臨終的時候,提到以色列族將來要出埃及,並為自己的骸骨留下遺命。
\par
23 
摩西生下來,他的父母見他是個俊美的孩子,就因著信,把他藏了三個月,並不怕王命。
\par
24 
摩西因著信,長大了就不肯稱為法老女兒之子。
\par
25 
他寧可和神的百姓同受苦害,也不願暫時享受罪中之樂。
\par
26 
他看為基督受的凌辱比埃及的財物更寶貴,因他想望所要得的賞賜。
\par
27 
他因著信,就離開埃及,不怕王怒;因為他恆心忍耐,如同看見那不能看見的主。
\par
28 
他因著信,就守(或作:立)逾越節,行灑血的禮,免得那滅長子的臨近以色列人。
\par
29 
他們因著信,過紅海如行乾地;埃及人試著要過去,就被吞滅了。
\par
30 
以色列人因著信,圍繞耶利哥城七日,城牆就倒塌了。
\par
31 
妓女喇合因著信,曾和和平平的接待探子,就不與那些不順從的人一同滅亡。
\par
32 
我又何必再說呢?若要一一細說,基甸、巴拉、參孫、耶弗他、大衛、撒母耳,和眾先知的事,時候就不夠了。
\par
33 
他們因著信,制伏了敵國,行了公義,得了應許,堵了獅子的口,
\par
34 
滅了烈火的猛勢,脫了刀劍的鋒刃;軟弱變為剛強,爭戰顯出勇敢,打退外邦的全軍。
\par
35 
有婦人得自己的死人復活。又有人忍受嚴刑,不肯苟且得釋放(原文是贖),為要得著更美的復活。
\par
36 
又有人忍受戲弄、鞭打、捆鎖、監禁、各等的磨煉,
\par
37 
被石頭打死,被鋸鋸死,受試探,被刀殺,披著綿羊山羊的皮各處奔跑,受窮乏、患難、苦害,
\par
38 
在曠野、山嶺、山洞、地穴,飄流無定,本是世界不配有的人。
\par
39 
這些人都是因信得了美好的證據,卻仍未得著所應許的;
\par
40 
因為神給我們預備了更美的事,叫他們若不與我們同得,就不能完全。


\bye

