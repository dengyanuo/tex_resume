\input ../header/example_tex01.tex
\FFrh \baselineskip = 13pt
\parskip 0.3 em


\centerline{  \FFbg
腓 立 比 書 4 - Chinese Union Version (Traditional) }

\par
4 
我所親愛、所想念的弟兄們,你們就是我的喜樂,我的冠冕。我親愛的弟兄,你們應當靠主站立得穩。
\par
2 
我勸友阿爹和循都基,要在主裡同心。
\par
3 
我也求你這真實同負一軛的,幫助這兩個女人,因為他們在福音上曾與我一同勞苦;還有革利免,並其餘和我一同做工的,他們的名字都在生命冊上。
\par
4 
你們要靠主常常喜樂。我再說,你們要喜樂。
\par
5 
當叫眾人知道你們謙讓的心。主已經近了。
\par
6 
應當一無罣慮,只要凡事藉著禱告、祈求,和感謝,將你們所要的告訴神。
\par
7 
神所賜、出人意外的平安必在基督耶穌裡保守你們的心懷意念。
\par
8 
弟兄們,我還有未盡的話:凡是真實的、可敬的、公義的、清潔的、可愛的、有美名的,若有甚麼德行,若有甚麼稱讚,這些事你們都要思念。
\par
9 
你們在我身上所學習的,所領受的,所聽見的,所看見的,這些事你們都要去行,賜平安的神就必與你們同在。
\par
10 
我靠主大大的喜樂,因為你們思念我的心如今又發生;你們向來就思念我,只是沒得機會。
\par
11 
我並不是因缺乏說這話;我無論在甚麼景況都可以知足,這是我已經學會了。
\par
12 
我知道怎樣處卑賤,也知道怎樣處豐富;或飽足,或飢餓;或有餘,或缺乏,隨事隨在,我都得了祕訣。
\par
13 
我靠著那加給我力量的,凡事都能做。
\par
14 
然而,你們和我同受患難原是美事。
\par
15 
腓立比人哪,你們也知道我初傳福音離了馬其頓的時候,論到授受的事,除了你們以外,並沒有別的教會供給我。
\par
16 
就是我在帖撒羅尼迦,你們也一次兩次的打發人供給我的需用。
\par
17 
我並不求甚麼餽送,所求的就是你們的果子漸漸增多,歸在你們的賬上。
\par
18 
但我樣樣都有,並且有餘。我已經充足,因我從以巴弗提受了你們的餽送,當作極美的香氣,為神所收納、所喜悅的祭物。
\par
19 
我的神必照他榮耀的豐富,在基督耶穌裡,使你們一切所需用的都充足。
\par
20 
願榮耀歸給我們的父神,直到永永遠遠。阿們!
\par
21 
請問在基督耶穌裡的各位聖徒安。在我這裡的眾弟兄都問你們安。
\par
22 
眾聖徒都問你們安。在該撒家裡的人特特的問你們安。
\par
23 
願主耶穌基督的恩常在你們心裡!


\bye

